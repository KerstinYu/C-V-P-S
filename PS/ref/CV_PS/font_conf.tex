% %\renewcommand*\familydefault{\sfdefault}
% Only if the base font of the document is to be sans serif
% to set the default font; use '\sfdefault' for the default sans serif font, '\rmdefault' for the default roman one, or any tex font name

%\renewcommand{\sfdefault}{\rmdefault}
%\renewcommand{\rmdefault}{\sfdefault}


\ifxetex
\usepackage{fontspec}
\defaultfontfeatures{Ligatures=TeX} % To support LaTeX quoting style
%   \setromanfont{Garamond Premier Pro}
\setromanfont{Adobe Garamond Pro}
%\setsansfont{Droid Sans}
\setsansfont{Myriad Pro}
\else
\usepackage[utf8]{inputenc}
%%%%%%%%%%%%%%%%%%%%%%%%%%%%%%%%%%%%%%%%%%%%%%%%%%%
% sans fonts

%\usepackage{droid}
%\usepackage[lf]{venturis} %% lf option gives lining figures as default;
%\usepackage{arev}
%\usepackage[default]{cantarell} %% Use option "defaultsans" to use cantarell as sans serif only
%\usepackage{libertine}
%\usepackage[default]{comfortaa}
%\usepackage{DejaVuSans}
%\usepackage{dejavu}
%\usepackage{DejaVuSansCondensed}
%\usepackage[math]{iwona}
%\usepackage[defaultsans]{droidsans}
%\usepackage[scaled]{helvet}
%\usepackage[default]{lato}
%\usepackage[sfdefault]{merriweather} %% Option 'black' gives heavier bold face
%\usepackage[black]{merriweather} %% Option 'black' gives heavier bold face
%\usepackage{merriweather} %% Option 'black' gives heavier bold face
%\usepackage[default,osfigures,scale=0.95]{opensans}
%\usepackage[defaultsans,osfigures,scale=0.95]{opensans}
%\usepackage{paratype}
%\usepackage{PTSansCaption}
%\usepackage{PTSerifCaption}
%\usepackage[default]{sourcecodepro}
%\usepackage[default]{sourcesanspro}
%\usepackage{tgadventor}
%\usepackage{tgheros} %enchanced HELVETICA font
%\usepackage[lf]{venturis} %% lf option gives lining figures as default;

\renewcommand*\familydefault{\sfdefault} %% Only if the base font of the document is to be sans serif


%%%%%%%%%%%%%%%%%%%%%%%%%%%%%%%%%%%%%%%%%%%%%%%%%%%
%% change the font for the title: roman
%\documentclass[roman]{moderncv}

%%%%%%%%%%%%%%%%%%%%%%%%%%%%%%%%%%%%
%\usepackage[sc]{mathpazo}
%\usepackage{mathptmx}
%\usepackage{tgschola}
%\usepackage{tgbonum}
%\usepackage{tgtermes}
%\usepackage[black]{merriweather} %% Option 'black' gives heavier bold face
%\usepackage{merriweather} %% Option 'black' gives heavier bold face
%\usepackage[default]{droidserif}
%\usepackage{DejaVuSerif}
%\usepackage{dejavu}
%\renewcommand*\rmdefault{dayroms}
%\usepackage{bookman}
%\usepackage{librebaskerville}
%\usepackage[lining,scaled=.95]{ebgaramond} % The default is oldstyle numbers.
%\usepackage{accanthis}
%\usepackage{mathpazo}
%%%%%%%%%%%%%%%%%%%%%%%%%%%%%%%%%%%%

\fi

%%%%%%%%%%%%%%%%%%%%%%%%%%

% change the font for the entire document

%\documentclass{moderncv}
%\usepackage{mathpazo}


%\renewcommand{\rmdefault}{\sfdefault}

%%%%%%%%%%%%%%%%%%%%%%%%%%%%%%%%%%%%%%%%%%%%%%%%%%%
% change the font for some section,
% \namefont \titlefont \addressfont \sectionfont \subsectionfont \hintfont \quotefont
%\renewcommand*\namefont{\fontfamily{pzc}\fontsize{40}{48}\selectfont}
%\renewcommand*\titlefont{\fontfamily{pzc}\fontsize{20}{24}\selectfont}
%\renewcommand*\addressfont{\fontfamily{pzc}\selectfont}
%\renewcommand*\sectionfont{\fontfamily{pzc}\fontsize{20}{24}\selectfont}


%%%%%%%%%%%%%%%%%%%%%%%%%
%If you want to change the font for some particular argument(s) of,
%for example, the \cventry command you can define your own command to
%include the necessary redefinition(s); here's an example in which I
%defined a \Mycventry command using Zapf Chancery for the
%last argument of \cventry:

% define the font
%\newcommand\Mycventry[6]{%
%  \cventry{#1}{#2}{#3}{#4}{#5}{\fontfamily{pzc}\selectfont#6}}

% use the font in the body
%\Mycventry{2012}{Title}{Institute}{City}{}{\fontfamily{pzc}\selectfont \lipsum[2]}
%\cventry{2012}{Title}{Institute}{City}{}{\lipsum[2]}

%%%%%%%%%%%%%%%%%%%%%%%%%%%%%%%%%%%%%%%%%%%%%%%%%%%


%%%%%%%%%%%%%%%%%%%%%%%%%%%%%%%%%%%%%%%%%%%%%%%%%%%
%% change the font for the title
%\documentclass[roman]{moderncv}
%\usepackage{mathpazo}
%%%%%%%%%%%%%%%%%%%%%%%%%%

% change the font for the entire document

%\documentclass{moderncv}
%\usepackage{mathpazo}
%\renewcommand{\sfdefault}{\rmdefault}

%%%%%%%%%%%%%%%%%%%%%%%%%%%%%%%%%%%%%%%%%%%%%%%%%%%
% change the font for some section,
% \namefont \titlefont \addressfont \sectionfont \subsectionfont \hintfont \quotefont
%\renewcommand*\namefont{\fontfamily{pzc}\fontsize{40}{48}\selectfont}
%\renewcommand*\titlefont{\fontfamily{pzc}\fontsize{20}{24}\selectfont}
%\renewcommand*\addressfont{\fontfamily{pzc}\selectfont}
%\renewcommand*\sectionfont{\fontfamily{pzc}\fontsize{20}{24}\selectfont}


%%%%%%%%%%%%%%%%%%%%%%%%%
%If you want to change the font for some particular argument(s) of,
%for example, the \cventry command you can define your own command to
%include the necessary redefinition(s); here's an example in which I
%defined a \Mycventry command using Zapf Chancery for the
%last argument of \cventry:

% define the font
%\newcommand\Mycventry[6]{%
%  \cventry{#1}{#2}{#3}{#4}{#5}{\fontfamily{pzc}\selectfont#6}}

% use the font in the body
%\Mycventry{2012}{Title}{Institute}{City}{}{\fontfamily{pzc}\selectfont \lipsum[2]}
%\cventry{2012}{Title}{Institute}{City}{}{\lipsum[2]}

%%%%%%%%%%%%%%%%%%%%%%%%%%%%%%%%%%%%%%%%%%%%%%%%%%%
