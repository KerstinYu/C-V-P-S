\section{Experience}

\subsection{Academic}
% usage: \cventry[spacing]{years}{degree/job title}{institution/employer}{localization}{optionnal: grade/...}{optional: comment/job description}

\cventry{2014--Present}{Undergraduate thesis, supervised by Prof. Lihan Chen, \emaillink{clh20000@gmail.com}}{\link[Center for Brain and Cognitive Sciences]{cbcs.pku.edu.cn}, Peking University}{\textsc{Beijing}}{}{Investigated the overall direction perception of point light walkers and circular scattered dots with task-irrelevant face stimuli as background, which have three levels (angry, neutral or happy) of emotional valences and the role of perception style in this process. Specific methods are similar to my undergraduate research. Found that the overall direction perception of walkers showed a facing away effect, while a facing away effect was found for circular dots. Also, higher interpersonal reactivity influenced the directional perception of walkers more easily when a positive face valence background was presented. Subjects with higher social anxiety demonstrated a stronger facing bias than did the group with lower social anxiety. This pattern was not observed with random dot stimuli (without biological meaning). Overall, the data showed that perception of ambiguous walkers could be resolved by tactile input and modulated by higher social cognitive styles (empathy and social anxiety). Research hours: 400h}
%In a second experiment, we investigated the overall direction perception of PLWs and circular scattered dots with the background of on tasks irrelevant face stimuli background with different levelsof different emotional valences. We found that the overall direction perception of PLWs showed a facing away  effect, while a convergence facing away effect was found in circular dots case. Higher interpersonal reactivity has a tendency to enact  influenced moremore easily with the directional perception of PLWs while when a positive face valence background was presented. Subjects with higher social anxiety experiences more steady rivalry while no irrelevant face stimuli was present. demonstrated a stronger facing bias than did the group with lower social anxiety. This pattern was not observed with random dot stimuli (without biological meaning). Overall, the data showed that perception of ambiguous PLWs could be resolved by tactile input and modulated by higher social cognitive styles (empathy and social anxiety).

\cventry{2013--2014}{Undergraduate thesis, supervised by Prof. Lihan Chen, \emaillink{clh20000@gmail.com}}{Multisensory Research Lab, Department of Psychology, Peking University}{\textsc{Beijing}}{}{Simulated the feedback of point light walker's visual footfall, to investigate the role of this the tactile input in resolving directional depth (inward or outward) perception of ambiguous walkers in binocular rivalry through stereoscope, while the feedback is slightly alteredmodulated in temporal structure. Two different colored walkers were presented on each side of the screen slightly tilted symmetrically along azimuthal axis and therefore triggered depth perception. Tactile stimuli were presented on each corresponding ankles. Social anxiety level for each participant was assessed. Data showed direction of the dominant apparent motion corresponding to the different temporal structure of tactile stimuli influences that of walkers, and that the effect is stronger in observers with high empathy concern level. Research hours: 1000h}
%task irrelevant auditory cue influences direction recognition of point light walkers, triggeringintroducing perception bias. Observers with high social anxiety tended to see walkers as facing away more frequently than those with low social anxiety. Emotional faces, compared with neutral faces, more strongly interfered with the primary task. 
%Recent studies on multisensory pathway  showed that task irrelevant auditory cue influences direction recognition of point light walkers, triggeringintroducing perception bias (Brooks et. al.,2007). Observers with high social anxiety tended to see walkers as facing away more frequently than those with low social anxiety (CruysVan de Cruys et. al., 2013). Emotional faces, compared with neutral faces, more strongly interfered with the primary task (Attar e. al., 2010). In this study, we simulate the feedback of point light walker’s feet posture by touching the ground, to investigate the role of this the tactile inputtactile stimuli   in resolving direction perception of ambiguous point light walkers, through by  simulating the tactile feedback of a point light walker in binocular rivalry via  with stereoscope, while  the feedback is slightly alteredmodulated in temporal structure, we are interested in the effect of this tactile stimuli on direction perception of the forward vs. backward point light walkers.  The data showed that direction of the dominant apparent motion corresponding to the different temporal structure of tactile stimuli influences the direction perception of a point light walker, and that the effect is stronger in observers with high empathy concern level.  . In our second experiment, we investigated the overall direction perception of PLWs and circular scattered dots with the background of on tasks irrelevant face stimuli background with different levelsof different emotional valences. We found that the overall direction perception of PLWs showed a facing away  effect, while a convergence facing away effect was found in circular dots case. Higher interpersonal reactivity has a tendency to enact  influenced moremore easily with the directional perception of PLWs while when a positive face valence background was presented. Subjects with higher social anxiety experiences more steady rivalry while no irrelevant face stimuli was present. demonstrated a stronger facing bias than did the group with lower social anxiety. This pattern was not observed with random dot stimuli (without biological meaning). Overall, the data showed that perception of ambiguous PLWs could be resolved by tactile input and modulated by higher social cognitive styles (empathy and social anxiety).

\cventry{2012--2013}{Lab Assistant, supervised by Prof. Ming Bao, \emaillink{baoming@mail.ioa.ac.cn}}{Key Laboratory of Noise and Vibration Research, Institute of Acoustics, Chinese Academy of Sciences}{\textsc{Beijing}}{}{Collaboratively established Auditory Localization Lab, mainly responsible for psychophysical configurations and orderings for required hardware. Specifically, provided desired engineering parameters for the framework, speakers, chips etc.; selected and ordered high frequency professional monitor, laser pointer, LEDs, high precision potentiometer etc.}

\cventry{2012--2013}{Beijing Innovation Projects (independent project), supervised by Prof. Lihan Chen, \emaillink{clh20000@gmail.com}}{\link[Center for Brain and Cognitive Sciences]{cbcs.pku.edu.cn}, Peking University}{\textsc{Beijing}}{}{Two different colored point light walkers with opposite local walking directions (left or right) were presented on the center of the screen simultaneously. Walkers were masked by grey dynamic random noise dots and were projected through anachrome optical diopter glasses. Tactile stimuli simulating the visual footfall of walkers were presented on participant's corresponding index finger. They reported perceived dominant direction of visual walker with two pedal switch. Also tested while walkers were inverted. Found that task-irrelevant tactile stimuli could resolve binocular rivalry between ambiguous walkers under mask for upright walkers, suggesting the presence of tactile input effects high-level processing in visual modality. Research hours: 800h}

%Point light walker (PLW) has been widely applied to address the biological motion processing in the visual modality. Biological PLW has been recently employed in multisensory research, in which task-irrelevant auditory cues would bias the perception of walking direction of ambiguous PLWs (Brooks et al., 2007). In current study we asked whether the tactile inputs, which simulate the hitting grounds by foots, could affect the ambiguous directional perception of PLWs. We presented binocular rivalry PLWs with 13 red and 13 cyan dots, the two PLWs could be either upright or either inverted. One tap was always synchronized with the hitting of the visual foot of one PLW, and the other tap could lead (150 ms), synchronize or lag (150 ms) the other hitting foot of this PLW. Participants wore glasses with a red filter on the left eye and a cyan filter on the right eye during the course of the experiments and performed two tasks: (1) Motion Direction Determination ("Motion" task)-they were asked to press and hold the left or right foot switch which corresponds to the dominant perception of "left" or "right" direction of the PLWs ; (2) Visual Dots Number Comparison ("Number" task), in which the participants were required to discriminate whether red dots were more than cyan dots or vice versa. The results showed the synchronous inputs of tactile taps increased the delectability of motion directional discrimination in "upright" condition but it was not the case in the "invert" condition. Interestingly, for the "Number" task, the PLW synchronized with the tactile inputs were perceived to be more in dots. These findings suggest that tactile inputs, which consist of apparent motion, affect the visual apparent motion of PLW, this effect could not be reduced to a general shift in visuo-spatial attention, as reflected in the Number task.


\subsection{International Meetings}

\cventry{May, 2013}{Abstract and poster for the meeting, supervised by Prof. Lihan Chen, \emaillink{clh20000@gmail.com}}{Annual Meeting -- Vision Science Society 2013 (VSS 2013)}{\textsc{Naples, Florida}}{}{Title: Tactile inputs resolve the ambiguous perception of biological point light walkers. doi: \link[10.1167/13.9.190]{dx.doi.org/10.1167/13.9.190}}
%\cventry{May, 2013}{Abstract and poster for the meeting, supervised by Prof. Lihan Chen, \emaillink{clh20000@gmail.com}}{Annual Meeting -- Vision Science Society 2013}{\textsc{Naples, Florida}}{}{Title: Tactile inputs resolve the ambiguous perception of biological point light walkers.}
%\begin{itemize}
  %\item Abstract -- \textit{Journal of Vision} July 24, 2013 vol. 13 no. 9 article 190. doi: \link[10.1167/13.9.190]{dx.doi.org/10.1167/13.9.190}
  %\item Poster -- 23.427, Orchid Ballroom, Session: Motion: Biological motion.
%\end{itemize}

\cventry{July, 2013}{Poster presentation, supervised by Prof. Lihan Chen, \emaillink{clh20000@gmail.com}}{Annual Meeting -- The 9th Asia-Pacific Conference on Vision (APCV 2013)}{\textsc{Suzhou, Jiangsu}}{}{Title: Tactile temporal groupings bias perception of ambiguous point light walkers. doi: \link[10.1002/pchj.32]{dx.doi.org/10.1002/pchj.32}}

%\cventry{July, 2013}{Abstract for poster presentation}{The 9th Asia-Pacific Conference on Vision (APCV 2013)}{\textsc{Suzhou, Jiangsu}}{}{Supervised by Prof. Lihan Chen, \emaillink{clh20000@gmail.com}}{Tactile temporal groupings bias perception of ambiguous point light walkers.}
%\begin{itemize}
  %\item Poster -- 23.427, Bai Yun Hall, Session: Perceptual grouping, Visual search.
%\end{itemize}

%\cventry{Aug, 2013}{The Secretary-General}{The 23th World Philosophy Congress}{\textsc{Athens, Greece}}{Delegation of Peking University for 23th WCP}{Supervised by Prof. Jixing Li, \emaillink{pinghenglun@sohu.com}}{Responsible for visa application, scheduling, communication with Hellenic Committee, paper submission, publicity, socializing with related participants and security management.}

\cventry{Aug, 2013}{The Secretary-General}{The 23th World Philosophy Congress (23th WCP)}{\textsc{Athens, Greece}}{Delegation of Peking University for 23th WCP}{Responsible for official business, paper submission, publicity and socializing with related participants.}

%\cventry{Oct, 2012}{Translator}{International Congress for Traditional Chinese Medicine}{\textsc{Xinmi, Henan}}{Oral translation from English to Chinese during speech}{Supervised by Prof. Mark Harrison, \emaillink{mark.harrison@wuhmo.ox.ac.uk}}

\subsection{Vocational}
%\cventry{2013-2014}{Secretary}{China Center for Balance Theory Research, Peking University}{\textsc{Beijing}}{}{Supervised by Prof. Jixing Li, \emaillink{pinghenglun@sohu.com}}{we did this and that}
\cventry{July -- Sept, 2014}{Teacher (Developmental Psychology)}{Xinjiang Education Institute}{\textsc{Beijing}}{hired by Amang\"ul, \emaillink{542398786@qq.com}{}}{}

\cventry{May, 2013}{Intern}{Beijing Huilongguan (Psychiatric) Hospital}{\textsc{Beijing}}{supervised by Prof. Mingyi Qian, \emaillink{qmy@pku.edu.cn}{}}{}

\cventry{Nov, 2012}{Intern}{Weixiuyuan Kindergarten \& Pei--Chi School for Mentally Retarded Children}{\textsc{Beijing}}{supervised by Prof. Yanjie Su, \emaillink{yjsu@pku.edu.cn}{}}{}



%----------------------------------------------------------------------------------------
%	AWARDS SECTION
%----------------------------------------------------------------------------------------

