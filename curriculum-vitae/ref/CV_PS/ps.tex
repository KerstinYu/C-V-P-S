%------------------------------------
% general materials
% -----------------------------------
%PS:A candidate’s statement of academic purpose should discuss your academic and career objectives in a concise, sharply focused, and well-crafted essay. This statement must be completed online as part of the Online Graduate Student Application. Admissions committees are particularly interested in this statement, so it is considered a vital part of your application. Therefore, you should be as specific as possible in discussing your academic objectives and research interest. There is a 2,500-word limit for the statement of purpose on our online application. 
%
%Provide a brief statement of your scientific and professional interests and objectives. Include a description of your past accomplishments that are not evident from the examination of other documents submitted. Report, if applicable, on any research in progress. The statement must be written by the applicant in English. It must not be written in another language and translated for the applicant by another person.
%
%CalTech:Please prepare a business résumé that includes your employment history in reverse chronological order, with titles, dates, and whether you worked part-time or full-time. Your educational record should also be in reverse chronological order and should indicate dates of attendance and degree(s) earned. Other information appropriate to a business résumé is welcomed and encouraged. The résumé should not be more than one page in length (up to 50 lines). We provide a résumé template to follow in the online application.
%
%Yale:A 500-1000 word statement concerning your past work, preparation for the intended field of study, relevant background and interests, academic plans, and career objectives is required. It should be used to describe your reasons for applying to the particular Yale department or program. This statement may assist the admissions committee in evaluating your aptitude and motivation for graduate study.
%
%Sanford:Your Statement of Purpose should be concise, focused, and well written. It should describe succinctly your reasons for applying to the proposed program at Stanford, your preparation for this field of study, research interests, future career plans, and other aspects of your background and interests which may aid the admissions committee in evaluating your aptitude and motivation for graduate study. The Statement of Purpose must be no more than 8000 characters in length. This includes spaces in between words. it will be submitted as part of the online application.
%
%Duke:Your application must include a Statement of Purpose indicating your purposes and objectives in undertaking graduate study, your special interests and plans, and your strengths and weaknesses in your chosen field. Briefly describe any research projects or any independent research in which you have actively participated and indicate how this has influenced your career choice and desire to pursue graduate studies. If you have particular reasons for applying to Duke (e.g. you would like to work with a specific faculty member), please indicate these.
%
%internet:Those applying for graduate study in the US will generally need to provide a statement of purpose. This is a short essay that should demonstrate your critical intelligence and passion for the subject, your intellectual interests, and how your prior education has prepared you for the proposed course of study.
%
%Chicago:
%A personal statement of purpose is a required component of your application. Your statement should address your past work, preparation for the intended field of study, relevant background and interests, academic plans, and career. It should be used to describe your reasons for applying to the particular department or program. This statement may assist the admissions committee in evaluating your aptitude and motivation for graduate study. Beyond what is apparent from your transcripts, describe preparation for proposed program of study, including research projects in which you have participated or language and other academic training where appropriate.


%Harvard psychology on PS 
%ones that develop ideas, propose experiments, point to holes in the literature, and do these things with passion and excitement. These very general comments, which will certainly not capture every advisor’s perspective, or even the majority, can be distilled to a few essential ingredients, presented below as questions:
%? Why continue on with your education?  Why do you need to learn more? What skills, theories, and knowledge do you lack?
%? What are the kinds of discoveries and theories that sparked your interest in the chosen discipline? 
%? In graduate school, what kinds of questions do you hope to address? Why do you think that these questions are important?  Given the set of questions that you will focus on, what kinds of methods do you hope to apply? What skills do you bring forward as you enter graduate school and which skills do you hope to acquire?
%? What holes do you see in the current discipline [big picture stuff]? In what ways do you think that they can be addressed during your graduate career?
%? What kind of graduate environment are you looking for?  Are you particularly keen on working with one faculty advisor, and if so, why this particular person?  If you are leaning more toward a cluster of advisors, as well as the department more generally, why?  Hint: faculty are engaged by students who have read some of their work, have thought critically about it, and wish to develop some of the issues addressed.  Further, it helps with admissions to have one or more faculty championing your case.

%If you started college as a biology or physics major, for example, your less-than-optimal grades during your first two years of college may be perfectly understandable. Also, admissions committees will usually look more carefully at your recent grades than the grades you received two or three years ago.

%the ability to express ideas clearly, 
%a curious and creative intellect, 
%and mathematical and computational ability. Evidence of these skills in your application is especially helpful in the graduate admissions process.


% ----------------------------------------
%  summary
% ----------------------------------------
%academic and career objectives
%concise, sharply focused
%as specific as possible in academic objectives and research interest

%description of past accomplishments (not evident from other materials)
%Report, if applicable, on any research in progress

%concerning your past work, preparation for the intended field of study, relevant background and interests, academic plans, and career objectives 
%used to describe your reasons for applying to the particular Yale department or program
%assist the admissions committee in evaluating your aptitude and motivation 
%indicate how this has influenced your career choice and desire to pursue graduate studies

%%? Why continue on with your education?  Why do you need to learn more? What skills, theories, and knowledge do you lack?
%%? What are the kinds of discoveries and theories that sparked your interest in the chosen discipline? 
%%? In graduate school, what kinds of questions do you hope to address? Why do you think that these questions are important?  Given the set of questions that you will focus on, what kinds of methods do you hope to apply? What skills do you bring forward as you enter graduate school and which skills do you hope to acquire?
%%? What holes do you see in the current discipline [big picture stuff]? In what ways do you think that they can be addressed during your graduate career?
%%? What kind of graduate environment are you looking for?  Are you particularly keen on working with one faculty advisor, and if so, why this particular person?  If you are leaning more toward a cluster of advisors, as well as the department more generally, why?  Hint: faculty are engaged by students who have read some of their work, have thought critically about it, and wish to develop some of the issues addressed.  Further, it helps with admissions to have one or more faculty championing your case.

%%the ability to express ideas clearly, 
%%a curious and creative intellect, 
%%and mathematical and computational ability. Evidence of these skills in your application is especially helpful in the graduate admissions process.


% -------------
% Her basic interests are based on the measurement, analysis, and understanding of something invisible or complex. Her research interests include non-invasive monitoring, data mining, signal processing, wearable sensors, probabilistic models, social computing and digital fabrication.
% 
% Physiological signals, wearable sensors, Pattern Recognition, Probabilistic Models
% 
% The scientific understanding of the human mind is one of the greatest intellectual challenges of our time.  The exquisite complexity and sophistication of human intelligence has made it one of the most enticing and enduring scientific mysteries.
%  
%will freely
%choose to work long hours because it is fun and exciting
%there are frustrating times when your experiments are not working


%Understanding of the world is mainly a process of answering a series of `\emph{How}' questions, rather than hunting for the `\emph{Why}'s in the jungle. Deeply enchanted by the desire for knowledge, I devoted myself to learning the marvels of the world, acquiring ways to describe this knowledge and applying them back into life for a better well-being.

As an Uyghur, I was raised in Urumqi (Xinjiang, a province of Western China), an extremely conservative area with a different language and religion from the rest of China. As such, literature, especially scientific literature, is scarce and less documented in our native language. To get a better education, I decided to go to Changchun as a teenager to start my high school education.  Moving from Western China to East China, I quickly adapted myself to the new life style and learned proficient Mandarin. I gradually stood out among both local and ethnic minority students due to my academic performance. In 2010, I was the first student ever from Changchun Hope High School (with a history of over 60 years) to be accepted to Peking University, China's top university. 

%\setlength{\parfillskip}{0pt plus 1fil}
\setlength{\parindent}{15pt} % Default is 15pt.

The competitive learning environment in Peking University imposed more challenges than those in high school. In addition to learning the lessons for my major, I had to start to learn another foreign language, English, to equip myself to learn from influential papers in scientific journals. However, over time I adjusted both socially and academically, which is evidenced by the year-to-year GPA improvement (and increase in friends).

As for my research, I like to take a strict mathematical approach to solve scientific questions in a quantitative and measurable way. 

Early in my sophomore, I joined the Motor Control Lab where I learned psychological experiment designs and programming in MATLAB. I was first inspired by a paper on the role of auditory signals in resolving complex biological motion perception, and asked myself whether another sensory modality, such as tactile inputs could also have a similar effect. I contacted Dr. Lihan Chen from Multisensory Lab for help in designing and implementing the study. In his lab, I learned how to operate the tactile device from scratch. It was an arduous process to figure out the exact parameters such as various optical diopter  glasses, stereoscope positions, and the configurations of visual stimuli. In the end, I found the optimized parameters to simulate visual and tactile stimuli to measure the cross-modal effect. I spent around 800 research hours carrying out the experimental study with mostly self-taught MATLAB and data analysis. During this project, I learned how to scientifically pinpoint a specific problem and enjoyed solving the potential intriguing puzzles in this manner. This project was later expanded and won the Beijing Innovation Fund for Undergraduate Students to support the subsequent research. These innovative explorations were promising and were presented consecutively in Vision Science Society's Annual Conference in Naples, Florida 2013, and Asian Pacific Conference of Vision in Suzhou, China 2013.

My next step was to consolidate my findings with more paradigms and reveal the underlying cognitive mechanism in tactile-visual cross-modal interaction. Therefore, for my undergraduate thesis, I asked whether an individual's cognitive states could influence this modulation of tactile input to visual perception. I measured social anxiety and interpersonal reactivity including empathic concern and found that higher interpersonal reactivity influenced the directional perception of walkers more readily when a positive face valence background (happy face) was presented. Subjects with higher social anxiety also demonstrated a stronger facing bias. My thesis was highly appreciated by the evaluation committee and obtained the highest grade (A+). Now as a research assistant, I am starting to explore the neural mechanism with functional resonance magnetic imaging (fMRI).

%%%%%%%%%%%%%%%%%%%%%%%%%%%%%%%%%%%%%%%%%%%%%%%%%%%%%%%%%%%%%%
%%% Content below are school specific, revise before submit %%
%%%%%%%%%%%%%%%%%%%%%%%%%%%%%%%%%%%%%%%%%%%%%%%%%%%%%%%%%%%%%%

% KAUST AMCS
%During three years of research on cross-modal perception and also driven by great interest in Computer Science, I have developed an open-source library for biological motion research of over 8500 lines of MATLAB code. I was captivated by GNU/Linux and learned many programming languages and computation techniques such as pattern recognition and optimal control, and read a wide range of literature from G\"odel's Incompleteness Theorem to Turing's theory of universal computation to Minsky's Emotion Machine. I began to understand the exquisite complexity and sophistication of human intelligence and why a complete understanding of the human mind is one of the greatest intellectual challenges in the current age. To build and foster my computing skills, I attended many relevant courses including a collaborative advanced summer school seminar between Tsinghua University and New York University for graduate students on Bayesian Modeling of Perception. I also worked in the Motor Control lab in our department, implementing a virtual reality environment with Python, where an observer could wear a headset and use a joystick to target certain flying objects in the sky. This program was used to reduce reaction-time and operation delay in relevant tasks.  

% UCSC Psych|NYU | Dartmouth
During three years of research on cross-modal perception and also driven by great interest in Computer Science, I have developed an open-source library for biological motion research of over 8500 lines of MATLAB code. I was captivated by GNU/Linux and learned many programming languages and computation techniques such as pattern recognition and computer vision, and read a wide range of literature from leading journals recommended by several teachers on Bayesian modeling, motor control, decision-making and crossmodal perception. I began to understand the exquisite complexity and sophistication of human intelligence and why a complete understanding of the human mind is one of the greatest intellectual challenges in the current age. To build and foster my computing skills, I attended many relevant courses including a collaborative advanced summer school seminar between Tsinghua University and New York University for graduate students on Bayesian Modeling of Perception. I also worked in the Motor Control lab in our department, implementing a virtual reality environment with Python, where an observer could wear a headset and use a joystick to target certain flying objects in the sky. This program was used to reduce reaction-time and operation delay in relevant tasks.

%---------------
% UCSC CogSci
%During three years of research on cross-modal perception and also driven by great interest in Computer Science, I have developed an open-source library for biological motion research of over 8500 lines of MATLAB code. I was captivated by GNU/Linux and learned many programming languages and computation techniques such as pattern recognition and computer vision, and read a wide range of literature from leading journals recommended by several teachers on Bayesian modeling, motor control, problem solving and crossmodal perception. I began to understand the exquisite complexity and sophistication of human intelligence and why a complete understanding of the human mind is one of the greatest intellectual challenges in the current age. To build and foster my computing skills, I attended many relevant courses including a collaborative advanced summer school seminar between Tsinghua University and New York University for graduate students on Bayesian Modeling of Perception. I also worked in the Motor Control lab in our department, implementing a virtual reality environment with Python, where an observer could wear a headset and use a joystick to target certain flying objects in the sky. This program was used to reduce reaction-time and operation delay in relevant tasks.

%---------------
% UMN|USC|UCBoulder|Iowa|UIUC
%During three years of research on cross-modal perception and also driven by great interest in Computer Science, I have developed an open-source library for biological motion research of over 8500 lines of MATLAB code. I was captivated by GNU/Linux and learned many programming languages and computation techniques such as pattern recognition and computer vision, and read a wide range of literature from leading journals recommended by several teachers on Bayesian modeling, motor control, problem solving and crossmodal perception. I began to understand the exquisite complexity and sophistication of human intelligence and why a complete understanding of the human mind is one of the greatest intellectual challenges in the current age. To build and foster my computing skills, I attended many relevant courses including a collaborative advanced summer school seminar between Tsinghua University and New York University for graduate students on Bayesian Modeling of Perception. I also worked in the Motor Control lab in our department, implementing a virtual reality environment with Python, where an observer could wear a headset and use a joystick to target certain flying objects in the sky. This program was used to reduce reaction-time and operation delay in relevant tasks.

%---------------
% IUBloomington
%During three years of research on cross-modal perception and also driven by great interest in Computer Science, I have developed an open-source library for biological motion research of over 8500 lines of MATLAB code. I was captivated by GNU/Linux and learned many programming languages and computation techniques such as pattern recognition and optimal control, and read a wide range of literature from G\"odel's Incompleteness Theorem to Turing's theory of universal computation, from Minsky's Emotion Machine to Hofstadter's \textit{Strange Loop}. The more I read into these topics, the more I become attached to Indiana University, Bloomington. I began to understand the exquisite complexity and sophistication of human intelligence and why a complete understanding of the human mind is one of the greatest intellectual challenges in the current age. To build and foster my computing skills, I attended many relevant courses including a collaborative advanced summer school seminar between Tsinghua University and New York University for graduate students on Bayesian Modeling of Perception. I also worked in the Motor Control lab in our department, implementing a virtual reality environment with Python, where an observer could wear a headset and use a joystick to target certain flying objects in the sky. This program was used to reduce reaction-time and operation delay in relevant tasks.  

%---------------
% NWU
%During three years of research on cross-modal perception and also driven by great interest in Computer Science, I have developed an open-source library for biological motion research of over 8500 lines of MATLAB code. I was captivated by GNU/Linux and learned many programming languages and computation techniques such as pattern recognition and optimal control, and read a wide range of literature from leading journals recommended by several teachers on Bayesian modeling, motor control, problem solving and crossmodal perception. I began to understand the exquisite complexity and sophistication of human intelligence and why a complete understanding of the human mind is one of the greatest intellectual challenges in the current age. To build and foster my computing skills, I attended many relevant courses including a collaborative advanced summer school seminar between Tsinghua University and New York University for graduate students on Bayesian Modeling of Perception. I also worked in the Motor Control lab in our department, implementing a virtual reality environment with Python, where an observer could wear a headset and use a joystick to target certain flying objects in the sky. This program was used to reduce reaction-time and operation delay in relevant tasks. From this experience and getting familiar with other projects in the Motor Control Lab, I found motor control and machine learning really fascinating, and dreamed for an opportunity to join Prof. Kording, the advisor of my advisor in this lab.

%---------------
% MIT|UCSD.CSE
%During three years of research on cross-modal perception and also driven by great interest in Computer Science, I have developed an open-source library for biological motion research of over 8500 lines of MATLAB code. I was captivated by GNU/Linux and learned many programming languages and computation techniques such as pattern recognition and optimal control, and read a wide range of literature from G\"odel's Incompleteness Theorem to Turing's theory of universal computation to Minsky's Emotion Machine. I began to understand the exquisite complexity and sophistication of human intelligence and why a complete understanding of the human mind is one of the greatest intellectual challenges in the current age. To build and foster my computing skills, I attended many relevant courses including a collaborative advanced summer school seminar between Tsinghua University and New York University for graduate students on Bayesian Modeling of Perception. I also worked in the Motor Control lab in our department, implementing a virtual reality environment with Python, where an observer could wear a headset and use a joystick to target certain flying objects in the sky. This program was used to reduce reaction-time and operation delay in relevant tasks.  

%---------------
% CMU
%During three years of research on cross-modal perception and also driven by great interest in Computer Science, I have developed an open-source library for biological motion research of over 8500 lines of MATLAB code. The data used by this library was primarily acquired from Carnegie Mellon University Graphics Lab Motion Capture Database (MoCap). I received many help and guidance from Mocap lab members and dreamed that I would be studying at CMU for my graduate education. In addition, I was captivated by GNU/Linux and learned many programming languages and computation techniques such as pattern recognition and optimal control, and read a wide range of literature from G\"odel's Incompleteness Theorem to Turing's theory of universal computation to Anderson's ACT-R model. I began to understand the exquisite complexity and sophistication of human intelligence and why a complete understanding of the human mind is one of the greatest intellectual challenges in the current age. To build and foster my computing skills, I attended many relevant courses including a collaborative advanced summer school seminar between Tsinghua University and New York University for graduate students on Bayesian Modeling of Perception. I also worked in the Motor Control lab in our department, implementing a virtual reality environment with Python, where an observer could wear a headset and use a joystick to target certain flying objects in the sky. This program was used to reduce reaction-time and operation delay in relevant tasks.  

% ============================================================================================== 
% last paragraph
% ============================================================================================== 


% MIT
%My interest in social cognitive styles and computation urged me to look for an answer to this question: \textit{Whether and to what extent can computers recognize and emulate human emotions?} Intelligence is one important dimension in human's cognitive abilities while other dimensions such as emotions and creative-thinking also play an important role. However, intelligence has been the main concern in the majority of AI community, while emotions and creative thinking have largely been overlooked. After reading Picard's (1995) Affective Computing and some of her recent papers, I found a sense of passion and direction for my future scientific career. In order to pursue and develop this passion, I want to find a way to combine my cognitive psychology background with a deepened understanding of CS theories and AI techniques. I would like to address this question with a comprehensive approach, including probability-based computational modeling, facial-expression recognition, real-time visual facial expressions rendering, and wearable devices. I am prepared to learn any interesting new approaches and topics in this regard and would love for the opportunity to take on new challenges. Learning from the website and contacting several group members, I believe Dr. Picard's Affective Computing Group in MIT Media Lab could provide the best academic and social environment for my graduate study. If I am admitted to the PhD program in Brain and Cognitive Sciences at MIT, I am planning to join Dr. Picard's Affective Computing Group in MIT Media Lab.

%---------------
% UCB
%My interest in social cognitive styles and computation urged me to look for an answer to this question: \textit{How are emotional stimuli represented in the brain and could they be captured by computational models?} Intelligence is one important dimension in human's cognitive abilities while other dimensions such as emotions also play an important role. However, intelligence has been the main concern in the majority of AI community, while emotions have largely been overlooked. After reading some of recent papers by the Bishop Lab and the Gallant Lab, I found a sense of passion and direction for my future scientific career. In order to pursue and develop this passion, I want to find a way to strengthen my cognitive psychology background and learn computational modeling to a further extent. I would like to address this question with a comprehensive approach, including probability-based computational modeling, fMRI, system identification, and face processing. I am prepared to learn any interesting new approaches and topics in this regard and would love for the opportunity to take on new challenges. Learning from the lab website and making initial contact with the potential PIs, I believe Department of Psychology in UC Berkeley could provide the best academic and social environment for my graduate study.


%---------------
% CMU
%My interest in social cognitive styles and computation urged me to look for an answer to this question: \textit{How are emotional stimuli represented in the brain and could they be captured by computational models?} Intelligence is one important dimension in human's cognitive abilities while other dimensions such as emotions also play an important role. However, intelligence has been the main concern in the majority of AI community, while emotions have largely been overlooked. After finding out about ACT-R and reading some of the latest papers on the theory, I found a sense of passion and direction for my future scientific career. In order to pursue and develop this passion, I want to find a way to strengthen my cognitive psychology background and learn computational modeling to a further extent. For my graduate research, I would like to try to explore cognitive models on processes involving decision-making, creative thinking etc. in environments with complexities, such as the presence of affective facial expression or speech. What exactly is going on while people performing these complex tasks is an intriguing topic, and combining neuroimaging techniques (fMRI) and computational modeling could provide a clearer understanding. I am also interested in strategic games (I even wrote an emulator for the board game \textit{The Resistance} and a bot to play the game last year), since this is also an interesting way to study how people solve problems. I would like to address this question with a comprehensive approach, including probability-based computational modeling, fMRI, machine-learning, pattern recognition and face processing. I am prepared to learn any interesting new approaches and topics in this regard and would love for the opportunity to take on new challenges. I was excited to learn from several lab websites (ACT-R, CCBI, SIBR etc.) about their research interests and ongoing research, and tried making initial contact with the potential PIs. Namely, I would love to join Prof. Johh Anderson's group or other members of this group such as Prof. Cleotilde González, Christian Lebiere, Jungaa Moon. In addition, I have found Prof. Marcel Just's approach on problem-solving and machine-learning; Prof. David Klahir's PIER program on scientific reasoning very interesting and would love the opportunity to work with them. Therefore, I believe Department of Psychology in Carnegie Mellon University could provide the best academic and social environment for my graduate study.


%---------------
% NWU
%My interest in biological motions and computational neuroscience urged me to look for an answer to this question: \textit{How is behavior represented in the nervous system and to what extent could they be captured by computational models?} We are very good at acquiring new motor skills and generalizing some of them to different situations. But how exactly this change happens in the nervous system is controversial. After finding out about Bayesian modeling and reading some of the latest papers on the application of the theory, I found a sense of passion and direction for my future scientific career. In order to pursue and develop this passion, I want to find a way to strengthen my cognitive psychology background and learn computational modeling to a further extent. For my graduate research, I would like to try to explore cognitive models on processes involving motol control, decision-making, creative thinking etc. in an environment with complexities or uncertainties such as the presence of cognitive bias, affective facial expression, speech. What exactly is going on while people are performing these complex tasks is an intriguing topic, and combining electrophysiological experiments and computational modeling could provide a clearer understanding. I would like to address this question with a comprehensive approach, including probability-based computational modeling, behavioral experiments, fMRI, and pattern recognition. I am also interested in creative thinking, mostly logical paradoxes (Russel's Paradox, for example) since this is also an interesting way to study how people solve new complex problems. I am prepared to learn any interesting new approaches and topics in these regards and would love for the opportunity to take on new challenges. I was excited to learn from several of the lab websites (Prof. Kording's Bayesian Behavior Lab and Prof. Beeman's Cognitive Brain Mapping Group) about their research interests and ongoing research, and tried making initial contact with the potential PIs and some lab members. Namely, I would love to join Prof. Kording's group or Prof. Beeman's group if I am admitted. In addition, I have found Prof. Douglas Medin's approach on decision making and differences due to cultural background; Prof. Lance Rips' research on scientific reasoning very interesting and would love the opportunity to work with them. Therefore, I believe Department of Psychology in Northwestern University could provide the best academic and social environment for my graduate study.

%---------------
% UMN
%My interest in social cognitive styles and computational neuroscience urged me to look for an answer to this question: \textit{How is brain capable of complex cognitive tasks with the presence of social stimuli and to what extent could it be captured by computational models?} We are very good at acquiring new motor skills  generalizing some of them to different situations, and recognizing social semantics from facial expressions. But how exactly this change happens in the nervous system is controversial. After finding out about Bayesian modeling and reading some of the latest papers on the application of the theory, I found a sense of passion and direction for my future scientific career. In order to pursue and develop this passion, I want to find a way to strengthen my cognitive psychology background and learn computational modeling to a further extent. For my graduate research, I would like to try to explore cognitive models on processes involving memory, affective learning, motor skill acquisition, decision-making etc. in an environment with complexities or uncertainties such as the presence of cognitive bias, affective facial expression and speech. What exactly is going on while people are performing these complex tasks is an intriguing topic, and combining behavioral experiments and computational modeling could provide a clearer understanding. I would like to address this question with a comprehensive approach, including probability-based computational modeling, behavioral experiments, fMRI, and pattern recognition. I am prepared to learn any interesting new approaches and topics in these regards and would love for the opportunity to take on new challenges. I was excited to learn about several faculty members (Prof. Chad Marsolek, Prof. Paul Schrater) on their research interests and ongoing research from some of their recent papers, and tried making initial contact with the potential PIs and some lab members. I would love to join Prof. Marsolek's lab or Prof. Schrater group if I am admitted. I am also interested in Prof. Stephen Engel's approach with fMRI on visual plasticity and motion from occlusion; Prof. Daniel Kersten's research with ambiguous image data in Computational Vision Lab; Prof. Gordon Legge's apprach to object recognition with binocular vision and Prof. Wilma Koutstaal's emphasis on creativity. I would welcome the opportunity to work with them. Therefore, I believe Department of Psychology in University of Minnesota could provide the best academic and social environment for my graduate study.

%---------------
%USC
%My interest in social cognitive styles and computation urged me to look for an answer to this question: \textit{How are emotional stimuli processed in the brain and could the process be captured by computational models?} We are very good at complex or ambiguous pattern recognition tasks, while excelling at recognizing social semantics from facial expressions, which is yet a hard task for AIs. But how exactly this recognition happens in the nervous system is controversial. After finding out about Bayesian modeling and reading some of the latest papers on the application of the theory, I found a sense of passion and direction for my future scientific career. In order to pursue and develop this passion, I want to find a way to strengthen my cognitive psychology background and learn computational modeling to a further extent. For my graduate research, I would like to try to explore cognitive models on processes involving pattern recognition (face recognition in particular, decision-making, creative thinking etc. with the presence of complexities in the environment, such as affective facial expression or speech. What exactly is going on while people performing these complex tasks is an intriguing topic, and combining neuroimaging techniques (fMRI) and computational modeling could provide a clearer understanding. I would like to address this question with a comprehensive approach, including probability-based computational modeling, fMRI, machine-learning, pattern recognition and face processing. I am prepared to learn any interesting new approaches and topics in this regard and would love for the opportunity to take on new challenges. I was excited to learn from several lab websites (Image Understanding Lab, Emotion \& Cognition Lab and t-lab) about their research interests and ongoing research, and tried making initial contact with the potential PIs. I would love to join Prof. Irving Biederman's lab, Prof. Mara Mather lab or Prof. Bosco Tjanif's lab I am admitted. I am quite interested in Prof. Irving's approach with fMRI on face recognition and scene perception; Prof. Mather's research about the effect of emotion in decision making and Prof. Gordon Legge's topic about adaptive representation of objects and ocular-motor control. I am looking for an opportunity to work with them. Therefore, I believe Department of Psychology in University of Southern California could provide the best academic and social environment for my graduate study.

%---------------
% UCBoulder
%My interest in social cognitive styles and computation urged me to look for an answer to this question: \textit{How are emotional stimuli represented in the brain and could they be captured by computational models?} Intelligence is one important dimension in human's cognitive abilities while other dimensions such as emotions also play an important role. However, intelligence has been the main concern in the majority of AI community, while emotions have largely been overlooked. After finding out about Bayesian modeling and reading some of the latest papers on the application of the theory, I found a sense of passion and direction for my future scientific career. In order to pursue and develop this passion, I want to find a way to strengthen my cognitive psychology background and learn computational modeling to a further extent. For my graduate research, I would like to try to explore cognitive models on processes involving emotions, high-order social perception, decision making etc. What exactly is going on while people performing these complex tasks is an intriguing topic, and combining neuroimaging techniques (fMRI) and computational modeling could provide a clearer understanding. I would like to address this question with a comprehensive approach, including probability-based computational modeling, fMRI, machine-learning, pattern recognition and neural network analysis. I am prepared to learn any interesting new approaches and topics in this regard and would love for the opportunity to take on new challenges. I was excited to find out Cognitive and Affective Neuroscience (CANLab) and their research interests and ongoing research, and tried making initial contact with the potential PI. Namely, I would love to join Prof. Tor Wager's group. I am also quite interested in Prof. Randall O'Relly's approach on neural bases of higher-order perception; Prof. R. McKell Carter's research on pattern analysis and decision making and Prof. Matt Jones's methematical approach to representational learning. I am also looking forward to the opportunity to work with them. Therefore, I believe Department of Psychology in University of Colorado at Boulder could provide the best academic and social environment for my graduate study.


%---------------
% Iowa
%My interest in crossmodal perception and computational modeling urged me to look for an answer to this question: \textit{How is brain capable of complex cognitive recognition tasks with the presence of social context and to what extent could it be captured by computational models?} We are very good at complex or ambiguous pattern recognition tasks, while excelling at recognizing social semantics from facial expressions, which is yet a hard task for AIs. But how exactly this recognition happens in the nervous system is controversial. After finding out about Bayesian modeling and reading some of the latest papers on the application of the theory, I found a sense of passion and direction for my future scientific career. In order to pursue and develop this passion, I want to find a way to strengthen my cognitive psychology background and learn computational modeling to a further extent. For my graduate research, I would like to try to explore cognitive models on processes involving visual/auditory pattern recognition and decision-making etc. in an environment with complexities or uncertainties such as the presence of cognitive bias, scene context and speech. What exactly is going on while people are performing these complex tasks is an intriguing topic, and combining behavioral experiments and computational modeling could provide a clearer understanding. I would like to address this question with a comprehensive approach, including probability-based computational modeling, behavioral experiments and pattern recognition. I am prepared to learn any interesting new approaches and topics in these regards and would love for the opportunity to take on new challenges. I was excited to learn about Prof. Bob McMurray, Prof. Andrew Hollingworth and Prof. D. Windschitl on their research interests and ongoing research from some of their recent papers, and tried making initial contact with the potential PIs and some lab members. I would love to join any of their labs if I am admitted. I am quite interested in Prof. McMurray's approach with computational modeling on speech recognition; Prof. Hollingworth's research with scene perception and object recognition and Prof. Windschitl's apprach to decision making with probabilistic models. I am looking forward to the opportunity to work with them. Therefore, I believe Department of Psychology in University of Iowa could provide the best academic and social environment for my graduate study.

%---------------
% IUBloomington
%My interest in social cognitive styles and computation urged me to look for an answer to this question: \textit{Whether and to what extent can computers recognize and emulate human emotions?} Intelligence is one important dimension in human's cognitive abilities while other dimensions such as emotions and creative thinking also play an important role. However, intelligence has been the main concern in the majority of AI community, while emotions have largely been overlooked. After finding out about Bayesian modeling and reading some of the latest papers on the application of the theory, I found a sense of passion and direction for my future scientific career. In order to pursue and develop this passion, I want to find a way to combine my cognitive psychology background with a deepened understanding of CS theories and AI techniques. For my graduate research, I would like to try to explore computational cognitive models on processes involving facial expression recognition/emulation, action recognition etc. What exactly is going on while people performing these complex tasks is an intriguing topic, and combining neuroimaging techniques (fMRI) and computational modeling could provide a clearer understanding. I am also interested in strategic games (I even wrote an emulator for the board game \textit{The Resistance} and a bot to play the game last year), since this is also an interesting way to study how people solve problems and come up with novel ideas. I would like to address this question with a comprehensive approach, including probability-based computational modeling, fMRI, machine-learning, pattern recognition and face processing. I am prepared to learn any interesting new approaches and topics in this regard and would love for the opportunity to take on new challenges. I see myself as a university professor doing research to try and extend our understanding to answer that question.

%Having strong passion for cognitive sciences, IU Bloomington has since been my dream university. Not only she possesses cutting-edge facilities in both computer sciences with its supercomputing resources and cognitive psychology with its fMRIs, TMS, she also has the very leading scholars in cognitive sciences. There are several labs that I am especially interested. Namely, I am interested in Cognitive Computing Lab (CCL) directed by Dr. Michael Jones. Their AI-based learning technologies concerning large-scale semantic modeling. I am also interested in Cognitive NeuroImaging Lab directed by Dr. Sharlene D. Newman. Her projects involving the effect of emotions on problem-solving really captivated me. In addition, Dr. Chen Yu's Computational Cognition and Learning Lab has several projects in which I would love to work as a graduate student, such as his study on multimodal human-robot interaction with Nao Robot and dynamic coupled bodily actions with his Phasespace Motion Tracking System. Relating to my previous research, I am also interested in IU Social Neuroscience Lab directed by Dr. Aina Puce. Her approach to biological motions and social cognition with fMRI, EET and TMS could provide me with much deeper understanding in the neural basis for these higher-order cognitive skills. In the same aspect, I would love the opportunity to join Vision Lab directed by Dr. Jason M. Gold and keep on doing research on biological motions and facial expression recognition. Since I am also interested in decision making, I found the projects about statistical learning, visual feature extraction and decision making in Memory and Perception Lab directed by Rechard Shiffrin quite impressive as well as Dr. Robert M. Nosofsky's research involving relations between categorization and decision making. There are so many attractive research going on in IU Bloomington and I really cannot wait to dedicate myself to any of them. I am looking forward to join any of these interesting projects for my doctoral education.

%Moreover, I really appreciate the emphasis in diversity at IU Bloomington. Therefore, I believe the PhD program in Cognitive Sciences in Indiana University Bloomington could provide the best academic and social environment for my graduate study.


%---------------
% UCSD CogSci
%My interest in biological motions and computational modeling urged me to look for an answer to this question: \textit{How is brain capable of complex cognitive tasks across contexts with uncertainty and to what extent could it be captured by computational models?} We are very good at acquiring new motor skills (being able to generalize some of them to different situations) or expressing preference under ambiguity (recognizing social semantics from facial expressions). But how exactly this change happens in the nervous system is controversial. After finding out about Bayesian modeling and reading some of the latest papers on the application of the theory, I found a sense of passion and direction for my future scientific career. In order to pursue and develop this passion, I want to find a way to combine my cognitive psychology background with a deepened understanding of CS theories and AI techniques. For my graduate research, I would like to try to explore computational cognitive models on processes involving learning, decision-making etc. in an environment with complexities or uncertainties such as the presence of cognitive bias. What exactly is going on while people are performing these complex tasks is an intriguing topic, and I believe combining behavioral experiments and computational modeling could provide a clearer understanding. I would like to address this question with a comprehensive approach, including probability-based computational modeling, machine learning, pattern recognition, behavioral experiments, and fMRI. I am prepared to learn any interesting new approaches and topics in these regards and would love for the opportunity to take on new challenges. 

%I am very interested in Prof. Angela J. Yu's revolutionary context-sensitive active vision model which optimizes relative detection accuracy against sensorimotor cost rather than internal belief system. Her approach to decision-making with Bayesian modeling is also interesting to me. In this regard, I found Prof. Virginia de Sa's research combining machine learning and human learning really interesting. In addition, Prof. James D. Hollan's emphasis on pen-and-paper based human-computer interaction seems quite fascinating to me, as well as his action recognition projects. Relating to my previous research, I am also interested in Prof. Ayse Saygin's research on biological motions. Her approach to biological motions and social cognition with fMRI, EEG and TMS could provide me with much deeper understanding in the neural basis for these higher-order cognitive skills. There are so many attractive research going on in UC San Diego and I really cannot wait to dedicate myself to any of them. I am looking forward to join any of these interesting projects for my doctoral education.

%Having strong passion for cognitive sciences, UC San Diego has been my dream university. Moreover, I really appreciate the emphasis in diversity at UC San Diego. Therefore, I believe the PhD program in Cognitive Sciences in UC San Diego could provide the best academic and social environment for my graduate study.

% UCSD Psych
%My interest in social cognition styles and computational modeling urged me to look for an answer to this question: \textit{How is brain capable of complex cognitive tasks with the presence of social context and to what extent could it be captured by computational models?} We are very good decision making at complex or ambiguous environments. But how exactly this process happens in the nervous system is controversial. After finding out about Bayesian modeling and reading some of the latest papers on the application of the theory, I found a sense of passion and direction for my future scientific career. In order to pursue and develop this passion, I want to find a way to strengthen my cognitive psychology background and learn computational modeling to a further extent. For my graduate research, I would like to try to explore cognitive models on processes involving decision-making, emotions etc. in an environment with complexities or uncertainties such as the presence of cognitive bias, scene context and facial expressions. What exactly is going on while people are performing these complex tasks is an intriguing topic, and combining behavioral experiments and computational modeling could provide a clearer understanding. I would like to address this question with a comprehensive approach, including probability-based computational modeling, behavioral experiments and neuroimaging techniques. I am prepared to learn any interesting new approaches and topics in these regards and would love for the opportunity to take on new challenges. 

%I really enjoy Prof. John Serences' approach to decision making combining psychophysics, computational modeling, and neuroimaging techniques. I also found Prof. Piotr Winkielman's research on emotions involving other cognitive processes very interesting. In addition, I am very interested in Prof. Craig R. M. McKenzie's emphasis on Bayesian modeling for investigating choice and uncertainty. There are so many attractive research going on in UC San Diego and I really cannot wait to dedicate myself to any of them. I am looking forward to join any of these interesting projects for my doctoral education.

%Having strong passion for cognitive psychology, UC San Diego has been my dream university. Moreover, I really appreciate the emphasis in diversity at UC San Diego. Therefore, I believe the PhD program in Psychology in UC San Diego could provide the best academic and social environment for my graduate study.

% UCSD SCE
%My interest in social cognitive styles and computation urged me to look for an answer to this question: \textit{Whether and to what extent can computers recognize and emulate human emotions?} Intelligence is one important dimension in human's cognitive abilities while other dimensions such as emotions and creative-thinking also play an important role. However, intelligence has been the main concern in the majority of AI community, while emotions and creative thinking have largely been overlooked. After reading Picard's (1995) Affective Computing and some of her recent papers, I found a sense of passion and direction for my future scientific career. In order to pursue and develop this passion, I want to find a way to combine my cognitive psychology background with a deepened understanding of CS theories and AI techniques. I would like to address this question with a comprehensive approach, including probability-based computational modeling, facial-expression recognition, real-time visual facial expressions rendering, and wearable devices. I am prepared to learn any interesting new approaches and topics in this regard and would love for the opportunity to take on new challenges. 

%I am really fascinated by Prof. Gary Cottrell's GURU project on perceptual expertise \& face recognition. I have read Wang, P. et al. (2014) and Cottrell te al. (2011) in this regard, and enjoyed the appreciated the achievements of \textit{The Model}. I am also interested in Prof. Marian Stewart Bartlett's Computational Face Group in Machine Perception Lab. I would like to study and enhance her Computer Expression Recognition Toolbox to recognize emotions from facial expressions across cultures. In addition, projects such as active sensing with machine learning for decision-making in Computational \& Cognitive Neuroscience Lab directed by Prof. Angela Yu are related to my previous research experiences, and I would love an opportunity to work with her. There are so many attractive research going on in UC San Diego and I really cannot wait to dedicate myself to any of them. I am looking forward to join any of these interesting projects for my doctoral education.

%Having strong passion for cognitive psychology, UC San Diego has been my dream university. Moreover, I really appreciate the emphasis in diversity at UC San Diego. Therefore, I believe the PhD program in CSE in UC San Diego could provide the best academic and social environment for my graduate study.

% UIUC
%My interest in social cognition styles and computational modeling urged me to look for an answer to this question: \textit{What are the neural mechanisms underlying the processing of emotional information as social cues and to what extent could it be captured by computational models?} We are very good decision making in complex social context or ambiguous environments. But how exactly this process happens in the nervous system is controversial. After finding out about Bayesian modeling and reading some of the latest papers on the application of the theory, I found a sense of passion and direction for my future scientific career. In order to pursue and develop this passion, I want to find a way to strengthen my cognitive psychology background and learn computational modeling to a further extent. For my graduate research, I would like to try to explore cognitive models on processes involving emotions, decision-making etc. in an environment with complexities or uncertainties such as the presence of cognitive bias, social scene context and facial expressions. What exactly is going on while people are performing these complex tasks is an intriguing topic, and combining behavioral experiments and computational modeling could provide a clearer understanding. I would like to address this question with a comprehensive approach, including probability-based computational modeling, behavioral experiments and neuroimaging techniques. I am prepared to learn any interesting new approaches and topics in these regards and would love for the opportunity to take on new challenges. 

%I really enjoy Prof. Florin Dolcos' approach to social cognition and affective neuroscience combining behavioral studies, neuroimaging techniques, especially her project on issues related to the processing of emotional information in social context seemed of great significance to me. In addition, I found Prof. John E. Hummel's emphasis on probabilistic models and neurocomputational systems to investigate object recognition and explanation generation. I am also quite interested in Prof. Ranxiao Frances Wang's research about human computer interaction and four-dimentional spatial cognition. There are so many attractive research going on at University of Illinois at Urbana-Champaign and I am looking forward to join any of these interesting projects for my doctoral education.

%Having strong passion for cognitive psychology, UIUC has been my dream university. Moreover, I really appreciate the emphasis in diversity at UIUC. Therefore, I believe the PhD program in Psychology in UIUC could provide the best academic and social environment for my graduate study.

% NYU
%My interest in social cognition styles and computational modeling urged me to look for an answer to this question: \textit{What are the neural mechanisms underlying the processing of emotional information as social cues and to what extent could it be captured by computational models?} We are very good decision making in complex social context or ambiguous environments. But how exactly this process happens in the nervous system and how exactly is it affected by emotions are controversial. After finding out about Bayesian modeling and reading some of the latest papers on the application of the theory, I found a sense of passion and direction for my future scientific career. In order to pursue and develop this passion, I want to find a way to strengthen my cognitive psychology background and learn computational modeling to a further extent. For my graduate research, I would like to try to explore cognitive models on processes involving emotions, decision-making etc. in an environment with complexities or uncertainties such as the presence of cognitive bias, social scene context and facial expressions. What exactly is going on while people are performing these complex tasks is an intriguing topic, and combining behavioral experiments and computational modeling could provide a clearer understanding. I would like to address this question with a comprehensive approach, including probability-based computational modeling, behavioral experiments and neuroimaging techniques. I am prepared to learn any interesting new approaches and topics in these regards and would love for the opportunity to take on new challenges. 

%I really enjoy Prof. Liz Phelps' approach to social cognition and affective neuroscience combining behavioral studies, neuroimaging techniques, especially her project on issues related to the processing of emotional information and its effect on memory and learning. I am also quite interested in Prof. Nathaniel Daw's comprehensive approach to decision-making with an emphasis on Bayesian decision theory as well as Prof. Todo Gureckis' research approach to category learning and decision making in this regard. In addition, I found Prof. Laurence T. Maloney's projects on determining kinship based on facial cues very interesting as well as his research about speeded movement tasks under risk. Relating to my previous research on biological motion depth perception, I am quite attached to Prof. Michael S. Landy's research using various share form texture and contour. As for object perception and recognition, Prof. Denis Pelli's insights about the analogous nature of face recognition and word recognition. There are so many attractive research going on at the New York University and I am looking forward to join any of these interesting projects for my doctoral education.

%Having strong passion for cognitive psychology, New York University has been my dream university. Moreover, I really appreciate the emphasis in diversity at NYU. Therefore, I believe the PhD program in Psychology in NYU could provide the best academic and social environment for my graduate study.

% Dartmouth
My interest in social cognition styles and computational modeling urged me to look for an answer to this question: \textit{What are the neural mechanisms underlying the processing of emotional information as social cues and to what extent could it be captured by computational models?} We are very good at decision making in complex social context or ambiguous environments. But how exactly this process happens in the nervous system and how exactly is it affected by emotions are controversial. After finding out about Bayesian modeling and reading some of the latest papers on the application of the theory, I found a sense of passion and direction for my future scientific career. In order to pursue and develop this passion, I want to find a way to strengthen my cognitive psychology background and learn computational modeling to a further extent. For my graduate research, I would like to try to explore computational cognitive models on processes involving emotions, decision-making etc. in an environment with complexities or uncertainties such as the presence of cognitive bias, social scene context and facial expressions. What exactly is going on while people are performing these complex tasks is an intriguing topic, and combining behavioral experiments, neuroimaging techniques and computational modeling could provide a better understanding. I would like to address this question with a comprehensive approach, including probability-based computational modeling, behavioral experiments and neuroimaging techniques. I am prepared to learn any interesting new approaches and topics in these regards and would love for the opportunity to take on new challenges. 

I really enjoy Dr. Luke J Chang's approach to social cognition and affective neuroscience combining behavioral studies, neuroimaging techniques, computational modeling and economics theories, especially his research the use of social norms in decision making (Chang, L.J. et. al, 2011) and the role of anterior insula on relevant behavior (Kahnt, T. et. al, 2012). I am also quite interested in Prof. Nathaniel Daw's comprehensive approach to decision-making with an emphasis on probabilistic models. I am also interested in Prof. James V. Haxby's research topics about representational spaces and face perception using computational modeling for classification. In addition, I also found the Whalen Lab directed by Prof. Paul J. Whalen an interesting place to do research, combining findings from human  research with that of animals in understanding the role of amygdala, and their collaborative research with Prof. Bill Kelley on neural correlates of reward. There are so many attractive research going on at the Dartmouth College and I am looking forward to join any of these interesting projects for my doctoral education.

Having strong passion for cognitive psychology, Dartmouth College has been my dream school. Moreover, I really appreciate the emphasis in diversity at Dartmouth College. Therefore, I believe the PhD program in Psychology in Dartmouth College could provide the best academic and social environment for my graduate study.

% KAUST
%My interest in social cognitive styles and computation urged me to look for an answer to this question: \textit{Whether and to what extent can computers recognize and emulate human emotions?} Intelligence is one important dimension in human's cognitive abilities while other dimensions such as emotions and creative-thinking also play an important role. However, intelligence has been the main concern in the majority of AI community, while emotions and creative thinking have largely been overlooked. After reading about Affective Computing and a couple of other papers in Cognitive Science, I found a sense of passion and direction for my future scientific career. In order to pursue and develop this passion, I want to find a way to combine my cognitive psychology background with a deepened understanding of CS theories and AI techniques and herdened mathematical skills. I would like to address this question with a comprehensive approach, including probability-based computational modeling, machine learning, facial-expression recognition and real-time visual facial expressions rendering. I am prepared to learn any interesting new approaches and topics in this regard and would love for the opportunity to take on new challenges. Learning from the website and contacting the potential PIs, I believe Prof. David I. Ketcheson's Numerical Mathematics Group in KAUST could provide the best academic and social environment for my graduate study. In addition, I really love the mixture of Islamic and the Western cultures. If I am admitted to the MS/PhD program in AMCS at KAUST, I am planning to join Prof. David Ketcheson's Numerical Mathematics Group.



%---------------
% here is nothing

% here is nothing
%---------------
